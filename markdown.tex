\documentclass[]{elsarticle} %review=doublespace preprint=single 5p=2 column
%%% Begin My package additions %%%%%%%%%%%%%%%%%%%

\usepackage[hyphens]{url}


\usepackage{graphicx}
%%%%%%%%%%%%%%%% end my additions to header

\usepackage[T1]{fontenc}
\usepackage{lmodern}
\usepackage{amssymb,amsmath}
% TODO: Currently lineno needs to be loaded after amsmath because of conflict
% https://github.com/latex-lineno/lineno/issues/5
\usepackage{lineno} % add
\usepackage{ifxetex,ifluatex}
\usepackage{fixltx2e} % provides \textsubscript
% use upquote if available, for straight quotes in verbatim environments
\IfFileExists{upquote.sty}{\usepackage{upquote}}{}
\ifnum 0\ifxetex 1\fi\ifluatex 1\fi=0 % if pdftex
  \usepackage[utf8]{inputenc}
\else % if luatex or xelatex
  \usepackage{fontspec}
  \ifxetex
    \usepackage{xltxtra,xunicode}
  \fi
  \defaultfontfeatures{Mapping=tex-text,Scale=MatchLowercase}
  \newcommand{\euro}{€}
\fi
% use microtype if available
\IfFileExists{microtype.sty}{\usepackage{microtype}}{}
\usepackage[]{natbib}
\bibliographystyle{plainnat}

\ifxetex
  \usepackage[setpagesize=false, % page size defined by xetex
              unicode=false, % unicode breaks when used with xetex
              xetex]{hyperref}
\else
  \usepackage[unicode=true]{hyperref}
\fi
\hypersetup{breaklinks=true,
            bookmarks=true,
            pdfauthor={Dawid Genert, Agnieszka Ancypo, Sylwia Bech},
            pdftitle={Raport z danych pacjentów},
            colorlinks=false,
            urlcolor=blue,
            linkcolor=magenta,
            pdfborder={0 0 0}}

\setcounter{secnumdepth}{0}
% Pandoc toggle for numbering sections (defaults to be off)
\setcounter{secnumdepth}{0}


% tightlist command for lists without linebreak
\providecommand{\tightlist}{%
  \setlength{\itemsep}{0pt}\setlength{\parskip}{0pt}}







\begin{document}


\begin{frontmatter}

  \title{Raport z danych pacjentów}
    \author[]{%
  %
  }
  
      \cortext[cor1]{Corresponding author}
  
  \begin{abstract}
  
  \end{abstract}
  
 \end{frontmatter}

\section{WSTĘP}\label{wstux119p}

Rak płuc jest jedną z najczęstszych i najbardziej śmiertelnych chorób
nowotworowych na świecie. Według danych Światowej Organizacji Zdrowia
(WHO) stanowi on jedną z głównych przyczyn zgonów z powodu nowotworów, a
jego rozwój często wiąże się z długotrwałym narażeniem na czynniki
środowiskowe i styl życia, takie jak palenie papierosów,
zanieczyszczenie powietrza, spożycie alkoholu czy kontakt z substancjami
toksycznymi w miejscu pracy.

Celem niniejszej analizy jest zbadanie czynników, które mogą mieć wpływ
na ryzyko zachorowania na raka płuc oraz identyfikacja potencjalnych
zależności między nimi. Wykorzystany zbiór danych pochodzi z platformy
Kaggle i zawiera informacje dotyczące pacjentów, obejmujące: - wiek,
płeć, - ekspozycję na zanieczyszczenia powietrza, alkohol, palenie
papierosów, - czynniki genetyczne, choroby przewlekłe, dietę, otyłość, -
objawy takie jak kaszel z krwią, duszność, utrata masy ciała itp.

Analiza ma na celu zbadanie zależności między wymienionymi czynnikami a
występowaniem raka płuc, a także wskazanie zmiennych, które mogą w
największym stopniu wpływać na ryzyko zachorowania. W ramach pracy
postawiono następujące pytania badawcze: 1. Jakie czynniki najczęściej
współwystępują z występowaniem raka płuc?\\
2. Czy istnieją różnice między płciami w częstości zachorowań?\\
3. Czy czynniki środowiskowe (zanieczyszczenie, palenie, alkohol) mają
istotny wpływ?


\end{document}
